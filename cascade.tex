%\documentclass[varwidth=170mm, border={2cm 2cm 2cm 2cm}]{standalone}
\documentclass{article}\usepackage[margin=2cm]{geometry}

\usepackage{lipsum}
\usepackage{hyperref}
\usepackage{cascade}

\title{THOUGHTS ON EVOLUTIONARY CREATION}
\author{Tobi}

\begin{document}
\maketitle
\begin{cascade}
    20241123
    \stepin
    I wrote a Jupyter notebook that creates an initial pop of NNs and then mutates them
    \stepin
    \url{https://github.com/tob-asco/gp-for-image-classification-cnn.git}\\
    mutations don't have a tendency towards improvement
    \stepin
    because of slow NN-teaching, I could only test on small pop sizes
    \stepout
    there's a lot of parameters the EC-engineer needs to decide in mutations
    \stepin
    e.g. an NN convolutional layer comes with a lot of choices (e.g. kernel sizes)\\
    in a sense this shifts the NN-hyperparameter optimization to an EC-hyperparameter optimization\\
    because the EC-space is either tightly bound or even discrete each EC-hyperparameter needs separate care
    \stepin
    e.g. different built-in PyTorch loss functions need different target value encoding\\
    e.g. kernel sizes, paddings and dilations of convolutional blocks depend on each other and input size\\
    e.g. built-in PyTorch optimizers come with sub-parameters
    \stepin
    the "separate care" here is that one needs to understand those quite deeply\\
    the hierarchy of EC-hyperparameters makes muatation more intricate
    \stepout
    \stepout
    \stepout
    \stepout
    I want to start my first EC-run
    \stepin
    Why?
    \stepin
    Because I now want experience in the actual field (NNs were a dummy project)
    \stepout
    What?
    \stepin
    Should be catchy so the topic itself creates attention\\
    Must be possible on my machine alone\\
    Probably needs to be a simulational run
    \stepin
    So the CoDs's values (for an ind.) are determined by simulation (of the environment)
\end{cascade}
\begin{cascade}
    20240905
    \stepin
    I'll stick with Python all the way (GP and NNs)\\
    But I'm not fit with it, so I'll need an IDE --> VS Code
\end{cascade}
\begin{cascade}
    20240904
    \stepin
    I went through the first few courses of learnpytorch.io
    \stepin
    got a sufficient overview of PyTorch 
    \stepin
    it is a flexible enough non-code-intensive python library that I'll use\\
    also I was able to move heavy computation to my GPU
    \stepout
    \stepout
    Now I want to create my first GP-run!
    \stepin
    TODOs:
    \stepin
    - find a way to represent an NN individual
    \stepin
    the representation should be focused on easy-access hyperparameters\\
    maybe a class where the constructor takes all hyperparameters as variable inputs
    \stepout
    - define the fitness measure
    \stepin
    + high accuracy on test data\\
    + small amount of network parameters (= number of weights)\\
    + quick learning (i.e. a steep learning curve in the beginning)
    \stepout
    - define the selection
    \stepin
    the 2019 Zhang paper uses a derivative of binary tournament selection
    \stepout
    - define crossover\\
    - mutation?\\
    - create an initial population \\
    - define weight initialization\\
    - choose which PyTensor layers individuals may use (= GP function set)\\
    - CoDs?
    \stepout
    \stepout
    Also, Laura recommended a nice book to simulate nature
    \stepin
    Daniel Shiffman's "Nature of Code"
    \stepin
    "Simulating Natural Systems with JavaScript"
    \stepout
    Possibly this could be used to create some environments for my EC-runs!
    \stepin
    recall that EC-runs need complex environments to obtain a high dimensional space of CoDs
\end{cascade}
\begin{cascade}
    20240812
    \stepin
    Ok I finished a first read of Nielsen19 "Neural Nets \& Deep Learning"
    \stepin
    a well-written, quite basic, little hands-on introduction to (deep \& convoluted) NNs
    \stepout
    Now I want to find the most efficient source code that offers maximal flexibility
    \stepin
    flexibility: hyperparameters should all be choosable upon NN definition
    \stepin
    net dimensions and kinds of layers\\
    activation functions\\
    cost functions\\
    parameter optimization techniques
    \stepin
    SGD, Hessian / momentum-based
    \stepout
    learning parameters
    \stepin
    $\eta$, epoch-count (resp. early stoppings)
    \stepout
    \stepout
    Nielsen's code is superseded because the core library "Theano" is unmaintained and integrated into "PyTensor"
    \stepin
    I tried to update the code to use PyTensor instead but the functions have changed too much to understand
    \stepout
    I asked Silvano, he recommends learning PyTorch
\end{cascade}
\begin{cascade}
    20240810
    \stepin
    A little interlude thought on why sex is crucial in Nature's creations
    \stepin
    otherwise, why would we find babies so cute\\
    otherwise, why would sex be such a satisfying activity\\
    otherwise, why would love be such a dominant feeling\\
    otherwise, why would the peacock have such good-looking feathers
\end{cascade}
\begin{cascade}
    20240722
    \stepin
    I'd like to get started with programming\\
    What are examples of CoD-problems?
    \stepin
    control engineering - Regelungstechnik\\
    to find methods against global warming\\
    building human prostheses or body add-ons
    \stepin
    create novel ways to communicate with the human body\\
    "body add-ons" refers to gadgets that enhance the human functions
    \stepin
    like glasses, but not only to mitigate shortcomings but also for cool extras
    \stepout
    this requires a realistic simulation of physics and the body
    \stepout
    create actual animals that serve additionally choosable purposes
    \stepin
    "Any stupid boy can crush a bug but all the profs in the world cannot make one"\\
    this is like extending Nature's Evolution with human-choosable CoDs\\
    I'm hoping to apply this against Global Warming, again
    \stepout
    composing songs
    \stepin
    possible CoDs: not catchy, no surprise, unknown rhythm, too repetitive
    \stepout
    find energy-storage options
    \stepin
    one could reduce to "chemical energy-storage"
    \stepout
    find energy-harvest options
    \stepin
    i.e. find sources of energy that can be converted to electrical energy\\
    i.e. solutions to convert some free and existing energy into electrical energy\\
    I mean on the Earth, so we would have to simulate Earth for that...
    \stepout
    build an inverse microwave
    \stepin
    a gadget that cools stuff within seconds
    \stepout
    find vaccines, medication, therapy methods
    \stepin
    here CoDs would partly correspond to causes of human death
    \stepout
    define new techniques of human/goods transportation
    \stepin
    like cars, plains, hyperloop, public transport, ...
    \stepout
    \stepout
    Ok, now examples that I could actually start with
    \stepin
    minesweeper agent\\
    navigate through a labyrinth
    \stepout
    Maybe first start with Neural Nets
    \stepin
    because they are probably quicker to understand than all those simulators
    \stepin
    because the given examples need a lot of simulation
    \stepin
    because CoD-Evolution needs to simulate its "environment"
\end{cascade}
\begin{cascade}
    20240721
    \stepin
    Some philosophical aspects AGAINST fitness measures as the driving force of Evolution
    \stepin
    people love to quantify stuff/others, i.e. define a total order
    \stepin
    either through money (jobs), or through citations (science), vote-count (politics), GPA (students), ...\\
    and this attitude has been a pain of mine for a long time - I dislike it almost categorically\\
    a fitness function is just another instance, so I'm naturally opposed
    \stepout
    fitness functions need uncountably much information which creates room for errors
    \stepin
    to provide a continuous function, you would have to specify too much to consider each case separately\\
    so ceratin high-level paradigms are used to create this vast "landscape of fitness"\\
    however, usually, edge cases with low-effort-high-fitness develop, creating "gaps"
    \stepin
    examples will be quite political, e.g. tax loopholes, planned obsolescence, child-labour
    \stepout
    \stepout
    on the other hand, a CoD/heaviside is defined by one point
    \stepin
    we can see it as a fitness function, but highly discontinuous and almost everywhere constant\\
    in a sense, it is a collection of multiple binary fitness measures
    \stepin
    recall from the "Intro" book that a binary fitness measure is fatal
    \stepin
    again, multiple CoDs are not binary, only one CoD is binary
    \stepout
    \stepout
    \stepout
    as in 20240717: my Evolution reads "survival of the immune" instead of Darwin's SOTF
    \stepin
    don't try to maximise 1 perfect measure
    \stepin
    even if it were a (weighted) some of all susceptibilities to the CoDs
    \stepin
    because then even the global maximum might die from some irrelevantly-weighted CoD
    \stepout
    \stepout
    instead, try to balance yourself between the various CoDs
    \stepin
    "well-rounded"
    \stepout
    rephrased, my Evolution has a non-scalar, non-continuous measure
    \stepin
    and it is not high fitness, it is aiming for, but low non-fitness
    \stepin
    which is not equivalent in non-scalar functions
    \stepin
    as you would need to choose a norm thingy to translate
    \stepout
    \stepout
    \stepout
    it is similar to high-school GPA and university admission
    \stepin
    you might only care about the grades in the subjects that you want to keep studying\\
    but if you fail a single class, you'll simply not be admitted to ANY university for ANY subject
\end{cascade}
\begin{cascade}
    20240719
    \stepin
    Why CoD-prescribed problems are efficiently solved by Stacked Random Search (SRS)
    \stepin
    SRS is what I praised in the notes 20240705
    \stepin
    where N random candidates are separated into G "generations"/stacks\\
    first only N/G random candidates are produced - the first gen(eration)\\
    then new N/G candidates are created from the former generation
    \stepin
    here, we may apply selection - a feedback loop between each generation\\
    this is the crucial difference between Random Search and Evolutionary Search
    \stepout
    \stepout
    the question now: what kind of problems are best suited for this kind of solution creation?
    \stepin
    the problem should have multiple independent subproblems
    \stepin
    because of the separation/fragmentation into generations\\
    I call this the fragmentation of the problem into subproblems\\
    and I view each CoD as 1 fragment of the problem defined through the bundle of all CoDs
    \stepout
    solutions might then also split up into substructures - each concerned only with subproblems\\
    hence, such substructures can be detected in a generation and carried over to the next gen
    \stepout
    is Natural Evolution such a problem?
    \stepin
    I think so, yes\\
    e.g., solutions have to avoid various CoDs
    \stepin
    starving, being killed by others, dying from illnesses, suicide, dying from overpopulation, ........
    \stepout
    \stepout
    do we see such similar substructures across Nature's species?
    \stepin
    yes, again\\
    e.g. almost all species of plants are green because they use photosynthesis against the "starving" CoD
\end{cascade}
\begin{cascade}
    20240718
    \stepin
    my midnight tai chi walk sparked some thoughts:
    \stepin
    EVOLUTIONARY CREATION (EC) is the name of my goal and it is different from GP
    \stepin
    EC runs should only affect the environment, not HOW evolution evolves its candidates
    \stepin
    namely through the definition of CoDs\\
    not through genetic operators (GO)
    \stepin
    ideally, the GOs should be defined canonically
    \stepin
    so, no subjectivity involved\\
    "crossover" shall evolve from self-cloning\\
    mutation might be constructable from first principles
    \stepin
    like the 2nd law of thermodynamics
    \stepout
    \stepout
    \stepout
    \stepout
    EC doesn't search for a global extremum, rather it searches for immunity
    \stepin
    you can view CoDs as a negative fitness measure, but it is highly discontinuous
    \stepin
    because each CoD has a heaviside function associated to it\\
    either you pass a CoD (survival = good fitness) or not (death = worst fitness)
    \stepout
    \stepout
    EC tries to create a trend towards creativity 
    \stepin
    creativity is encouraged if 2 requirements of development are met:
    \stepin
    R1. the rules have to be crystal clear
    \stepin
    I feel like CoDs are more clear than "maximise this smooth function"
    \stepout
    R2. the solution space should be left as unconstrained as possible, in the best case not even definable
    \stepin
    this is why we do not want to pre-impose any concrete GOs\\
    this is why a representation of variable-length bit-strings is my preferred choice
    \stepout
    \stepout
    \stepout
    \stepout
    if, in a GP run, GOs are highly specific, it is more an optimization than creation
\end{cascade}
\begin{cascade}
    20240717
    \stepin
    Where is evolution taking us?\\
    towards better?
    \stepin
    is the human design very good?
    \stepin
    I would love to have wings, e.g.\\
    also, I would prefer not to have been born with an appendix\\
    breathing under water would also be cool\\
    how about running as fast as a jaguar\\
    or having some sort of wheel construction instead of two feet\\
    what about seeing UV to know better when to protect from sunlight\\
    how about having some sensors to feel infections without measuring\\
    I would love to be a little less lazy\\
    why do I get addicted to stuff that doesn't do me good in the first place?\\
    why aren't people always happy?\\
    why aren't people more peaceful?
    \stepin
    actually, there might be a trend towards this\\
    I mean, compare to (movies of) the Middle Age
    \stepout
    \stepout
    are the primitive bacteria alive today as good as 4B years evolution would give if there were a trend towards better?
    \stepout
    towards more or less co-dependent?
    \stepin
    co-dependence is referring to a species' dependence on another species\\
    I feel like complex species almost always depend on others
    \stepin
    e.g. lions depends on antelopes\\
    e.g. don't some simple bacteria have zero co-dependence?\\
    however, e.g. viruses cannot survive without host (I think)
    \stepin
    so the inverse doesn't really hold
    \stepout
    as soon as energy from the sunlight is not sufficient, a co-dependence evolves, I think\\
    what about trees, they are pretty independent
    \stepin
    yet, they are pretty complex, no?
    \stepout
    \stepout
    the very first species ever, by definition, had no co-dependence
    \stepin
    so, indeed, relative to this example, co-dependence cannot decrease
    \stepout
    \stepout
    towards more complex?
    \stepin
    this actually sounds the most promising\\
    pro: in GP there's a lot of bloat evolving in later generations\\
    pro: everyone would agree that the human design is extremely complex\\
    pro: entropy increases with higher complexity
    \stepin
    because the amount of candidates (= microstates) increases with higher complexity (= macrovariable)
    \stepout
    pro: all non-complex alive species may just have had a later starting point
    \stepout
    Death as the central ingredient in GP:
    \stepin
    because it is the central ingredient in Nature's evolution, in my opinion\\
    First, define a set of causes of death (CoDs)
    \stepin
    a CoD is one test that a given individual may either pass (survive) or fail (die)\\
    CoDs may be completely different from one another
    \stepin
    this is the entry point for the programmer to guide his GP run\\
    the programmer may create a CoD for each property that shall NOT be found in the "fittest" individuals
    \stepin
    AHAAA: a shift of perspective from "who is the fittest" to "who is not fit enough"
    \stepin
    this essentially remedies my complaint 6. from 20240714
    \stepout
    \stepout
    \stepout
    CoDs should come with a number, the CoD's risk
    \stepin
    the risk should directly correspond to the frequency with which the risk's CoD is used as a test on candidates
    \stepout
    \stepout
    Second, test the candidates on CoDs over time
    \stepin
    within a generation, the GP run randomly - weighed by the risks - tests candidates on some CoDs\\
    this will produce a diverse new gen that is immune to a diverse selection of CoDs\\
    the goal is to find a generation that is immune to (almost) all CoDs
    \stepin
    is this the ultimate goal of Evolution: immunity?
    \stepout
    \stepout
    a CoD can be used to integrate two different categories of tendencies into a GP run
    \stepin
    1. environment
    \stepin
    the difficulties posed by the environment can be phrased as CoDs\\
    in fact, that is the more obvious category
    \stepout
    2. goal
    \stepin
    what you want to achieve can be phrased through a CoD\\
    e.g. you may define that blockage of low frequencies in an analog circuit is a CoD\\
    e.g. you may define that letting high frequencies pass is also a CoD
    \stepout
    \stepout
    \stepout
    Self-replication as the source of all genetic operators
    \stepin
    in a perfect GP run, the programmer may not choose genetic operators, yet all known genetic operators may evolve, and more!
    \stepin
    I want to mimic Nature's evolution here, but I don't know how replication actually started!
    \stepin
    I want to say cloning, i.e. haploidy
    \stepin
    it seems so easy, almost canonical
    \stepout
    but I don't really know\\
    I'll go with it, because I don't have the necessary background/idea to replace it with
    \stepout
    the hope is that cloning shall be superseded as a replication mechanism in favour of a better, self-developed mechanism
    \stepin
    as in Nature - because cloning couldn't repair fatally mutated DNA as well as diploid reproduction
    \stepin
    e.g. with humans, the father-mother-reproduction is extremely safe
    \stepin
    as opposed to incest (which is as close to cloning as I have examples of)
    \stepout
    cf. the nice chapter on homologous DNA in the non-Koza GP introduction book
    \stepout
\end{cascade}
\begin{cascade}
    20240714
    \stepin
    I had some time near Harlaching in order to meditate over Darwin's Evolution and "survival of the fittest"\\[3pt]
    1. "fittest" sounds like there's exactly ONE survivor/species and all the others are dead
    \stepin
    plain wrong, of course\\
    right now there is a myriad of living (hence "fittest") species on Earth \\
    there is also insane cross-dependence between them \\
    so try to stop thinking of a global maximum in evolution
    \stepout
    2. Darwin's SOTF doesn't capture the essence about how humans in the last 100 years have evolved
    \stepin
    here, the evolution of mind, intelligence, society, etc. is independent of survival\\
    I want a similar driving force for humans, their creativity and esprit
    \stepout
    3. there is not a single living entity that has survived more than 1% of earth's age
    \stepin
    so SOTF cannot be about individuals, right?\\
    so it should be about something like a species
    \stepin
    remember, a species is the set of organisms that mates with each other
    \stepout
    \stepout
    4. there is no reference to breeding
    \stepin
    would you say that a mosquito is "the fittest"?
    \stepin
    I mean, we're killing those beasts almost routinely
    \stepout
    no, but this is an example of a species that excels at their breeding rate
    \stepout
    5. I feel like Darwin's Evolution says that there is a long-time tendency towards BETTER
    \stepin
    I think this would only be true if the environment would have a long-time tendency towards harder
    \stepout
    6. there's too much about fitness when it should be more about DEATH
    \stepin
    because it's less about being super or the best, but more about avoiding death\\
    "survival of the death-avoiding" $\sim$ "survival of the surviving"
\end{cascade}
\begin{cascade}
    20240711
    \stepin
    crossover and replication are examples of genetic operators that have evolved
    \stepin
    e.g. cloning
    \stepin
    bacteria that couldn't clone itself just died\\
    bacteria that could would be dominant in the next generation
    \stepout
    e.g. male-female crossover
    \stepin
    higher organisms that would just clone itself were too susceptible to DNA mutation\\
    higher organisms that used male-female crossover mitigated this susceptibility and survived
    \stepout
    ChatGPT gives a lot of crossover examples that are different to human crossover
    \stepin
    Parthenogenesis, Polyembryony, Hermaphroditism, Assisted Reproductive Strategies (e.g. bee hives), Brood Parasitism
    \stepout
    \stepout
    so why should this be implemented by the genetic programmer, huh?\\
    I need to find a general, canonical framework that may create such genetic operators through program evolution
    \stepin
    maybe programs should "die" after some time
    \stepin
    DEATH - it is central in my opinion\\
    maybe there's exactly two ingredients: DEATH + MUTATION
    \stepin
    so, not so much about absolute and relative fitness?\\
    more about whether or not a certain fitness level is not passed - DEATH\\
    this goes towards the thought that evolution doesn't create better, but more complex/robust
    \stepout
    \stepout
    just like my reasoning from the "cloning" example above, this may create cloning\\
    cloning, in turn, may evolve into some sort of crossover\\
    ah, the beauty - it's time to start, damn' it..
    \stepin
    I'll wait and read a little more, this seems to be sparking lots of ideas
    \stepout
    \stepout
    "Mimicking evolution" is not well-specified because it can be done on many scales
    \stepin
    some examples that go from small scale to high scale:
    \stepin
    small: clone parts of a living creation\\
    larger: understand functionality of parts of living creations and replicate it elsewhere
    \stepin
    e.g. the bug's dirt repulsion system (from iENA) used in modern gadgets\\
    e.g. neural nets as inspired from brain's neurons
    \stepout
    larger: mimic the way that humans brain evolved from apes to today
    \stepin
    by copying homologous DNA recombination techniques in crossover
    \stepout
    larger: mimic the way that humans evolved from bacteria to today
    \stepin
    by copying mutation and fitness evaluation
    \stepout
    larger: mimic the way that entropy and death creates stuff
    \stepin
    death here involves a model of an environment together with natural laws\\
    at this scale we have no fitness measure, just mutation (=entropy) and death
    \stepout
    (largest?: mimic the way that entropy creates stuff)
    \stepin
    maybe death can be seen as entropy itself
    \stepout
    \stepout
    I feel a resemblance with the energy scale in fundamental physics
    \stepin
    the higher the energy the more fundamental the set of laws\\
    the higher the energy the more canonical and pure\\
    the higher the energy the more abstract and further away from experiment
\end{cascade}
\begin{cascade}
    20240710
    \stepin
    been reading 1998's "Genetic Programming An Introduction" by Banzhaf, Nordin, Keller, Francone \\
    I've put a handwritten list of interesting paragraphs inside the book\\
    the book is ok, but it is too theoretical
    \stepin
    this field is not about formality, it is about trial and error!
    \stepout
    there's too little about creation
    \stepin
    they say the boundaries of optimization and creation are "fuzzy"
    \stepin
    maybe, but still the distinction is central in my opinion
    \stepout
    a perfect GP setup should be as restrictive as a sheet of paper together with a pencil
    \stepin
    you should be able to write down Maxwell's equations\\
    you should be able to write down lyrics for your new song\\
    you should be able to sketch a scene from your last dream on it
    \stepout
    \stepout
    however they have taught me important aspects of biology
    \stepin
    e.g. a species is defined to be those individuals that mate with each other
    \stepin
    because two DNAs that are non-homologous cannot be recombined
    \stepin
    homologous DNAs means that their coarse structures agrees\\
    homology is needed to put upper bounds on the impact of recombination
    \stepout
    \stepout
    e.g. that sex's prime feature is to mitigate the impact of heavy mutation rather than being a source of diversity
    \stepin
    because heavy mutation in the father's DNA cannot be passed on to the kid if homology is demanded
    \stepout
    \stepout
    isn't creative/artistic work partly also like an internal "evolution of thought"?
    \stepin
    e.g. when writing lyrics:
    \stepin
    1. the composing artist has a topic in mind
    \stepin
    this topic greatly defines the fitness measure for the upcoming verses
    \stepout
    2. he begins by writing something\\
    3. he sings the verse in his head and evaluates how well this fits his intentions\\
    4. if ok, he goes to the next verse; if not sufficient, he might change parts of the verse
    \stepin
    this sounds like selection in the first case; mutation in the second case
    \stepout
    \stepout
    e.g. when sketching a picture from memory
    \stepin
    1. the artist draws a new part\\
    2. she looks at the newly emerged picture as a whole\\
    3. she evaluates the new picture against her memory of what it was before the new part\\
    4. she selects: either deletes the new part, or keeps going with it
\end{cascade}
\begin{cascade}
    20240706
    \stepin
    started reading Holland's "Adaption in Natural And artificial Systems"
    \stepin
    the new preface includes "I still find the 1975 preface surprisingly relevant. About the only change I would make would be to put more emphasis on improvement and less on optimization"
    \stepin
    I interpret as him slightly agreeing with me that evolution is not an optimization problem
    \stepin
    e.g. I think it's flawed to consider any of the currently living species a global maximum and the others not
    \stepin
    there is cross-dependence for example and this is just not reflected in a single number as fitness measure\\
    still, Koza's examples have totally convinced me already that a single fitness measure can greatly drive algo creation
    \stepin
    the point being that actual evolution in Nature is not as simple
    \stepout
    \stepout
    \stepout
    \stepout
    ok this book seems to be super dull, sorry!
    \stepin
    there's way to much formality, the field is not about formality, just like Koza says too
\end{cascade}
\begin{cascade}
    20240705
    \stepin
    started reading Koza's GP1\\
    he says "Nature creates structure over time by applying natural selection"\\
    and he says that nature uses none of the following principles that humans tend to do science with:
    \stepin
    correctness, consistency, justifiability, certainty, orderliness, parsimony, decisiveness
    \stepout
    so maybe Nature itself has been constructed also avoiding these principles \\
    so maybe Nature itself has been "evolved"
    \stepin
    the fitness measure may be whether or not the candidate universe lives
    \stepin
    e.g. has a cosmological constant such that the candidate universe's expansion is stable
    \stepout
    the different candidates may be encoded through different sets of laws
    \stepin
    so the "evolution of universes" in effect evolves its laws to be ever more stable, i.e. "living"
    \stepout
    this of course implies that multiple universes are "simultaneously" "present"\\
    furthermore it implies that these candidate universes have to have "sex"
    \stepin
    wuahh, weird.. let's say "candidate universe crossover"
    \stepout
    \stepout
    Randomness vs. Evolution
    \stepin
    Randomness is like Evolution with only 1 Gen
    \stepin
    Allowed size and complexity of the candidates dictates how big a (Gaussian) random distribution has to be to find a suitable candidate.
    \stepout
    Evolution is like "stacked"-Randomness, where each layer is 1 Gen
    \stepin
    Use the same size of population and divide it by after how many generations you want to terminate.\\
    Again - Gaussian or otherwise - randomly produce the first Gen, pick the best ones and build a new Gen "based on the best".\\
    Repeat for all Gens.
    \stepout
    Evolution sounds sooo much more efficient/better/clever
    \stepin
    Note that the two examples have the same amount of candidates to be generated\\
    and hence they share the amount of fitness evaluations as each candidate needs only be evaluated once (as it's fitness is constant in time)
\end{cascade}
\begin{cascade}
    20240704
    \stepin
    other possible GP applications
    \stepin
    battery design
    \stepin
    mainly a choice of material
    \stepout
    superconductor design
    \stepin
    mainly a choice of material
\end{cascade}
\begin{cascade}
    20240703
    \stepin
    today I'm watching Koza's video tapes that accompany his books:
    GP1
    \stepin
    I watched it yesterday\\
    the tree representation of a program (nodes = functions; leaves = input) is seminal for crossover
    \stepin
    you simply choose random nodes from the parents and exchange them to produce two children
    \stepout
    a great amount of examples, some of them are really amazing!
    \stepin
    I loved the inverse kinematic robot
    \stepout
    my criticism:
    \stepin
    you have to provide the functions, the evolution cannot create new atomic functions, i.e. building blocks
    \stepout
    \stepout
    GP2
    \stepin
    Automatically Defined Functions (ADFs) are the main topic
    \stepin
    they provide the framework for the algorithms to create their own subroutines\\
    the human has to specify the precise framework's structure
    \stepin
    e.g. the amount of ADFs\\
    e.g. the amount of input parameters per ADFs\\
    e.g. the hierarchy between multiple ADFs
    \stepin
    i.e. who is the outer function allowed to call the inner ones
    \stepout
    but in the very end he talks about evolving also this structure through evolution
    \stepin
    not sure what the take-away was here, though - was it better or worse?
    \stepout
    \stepout
    FRACTALITY i.e. SCALABILITY !! (I think)
    \stepout
    \stepout
    GP3
    \stepin
    he lists 16 (!) demands on the evolution and the evolved final algo
    \stepin
    1. Start only with "What needs to be done" (i.e. a high-lvl instruction)\\
    2. Tells us how to do it (huh?)\\
    3. Produces a computer program\\
    4. Automatically determines program size (huh?)\\
    5. Code reuse\\
    6. Parameterized reuse (why not in 5.?)\\
    7. Internal storage\\
    8. Iterations, loops, recursions\\
    9. Self-organization of hierarchies\\
    10. Automatically determines architecture (includes at least 9., right?)\\
    11. Wide range of constructs (huh?)\\
    12. Well-defined\\
    13. Problem-independent (big huh?)
    \stepin
    this might mean that e.g. both low-pass and band-pass filters can be evolved\\
    i.e. using the same evolution, population size, generation count
    \stepout
    ---------------------- up to here he feels like he checked them --------------------\\
    14. Wide applicability (vague..)\\
    15. Scalable (huh?)\\
    16. Human-competitive ("most important")
    \stepout
    he quotes Turing's seminal 1950 paper
    \stepin
    TODO: read it
    \stepout
    he explains what mutation and crossover are and how frequently they're employed
    \stepin
    mutation is a random growth of a deleted node; improbable (~1%)\\
    crossover is copying an existing node; very probable (~85%)
    \stepout
    architecture-altering operations add or delete:
    \stepin
    ADFs, their args, iterations, loops, recursions, memory
    \stepout
    automatic synthesis of analog electrical circuits
    \stepin
    I LOVE THIS IDEA\\
    there's an amazing animation on how trees dictate circuits in the video\\
    lots of examples of true AI
    \stepin
    it's amazing !!!
    \stepout
    \stepout
    "GP delivers a very high ratio of A-to-I" (artificial-to-input-intelligence)
    \stepout
    GP4
    \stepin
    it's about how broad GP solutions can be (cf. point 13. in GP3)
    \stepin
    1. it's easy and quick to change an evolution from one problem to another
    \stepin
    if the problems are in the same field, usually only change the fitness measure\\
    if the problems are from non-overlapping fields, additionally change nodes and leaves
    \stepout
    2. one can evolve algos with extra free variables to make it a multi-problem solution
    \stepin
    here, the fitness measure includes an integration over the free variables\\
    using the evolved algo w/ different free variable inputs yields solutions to different problems\\
    e.g. "general purpose controller" which they actually tried to patent
    \stepout
    \stepout
    there's lots of overlap between the other movies...\\
    "it is a system that works, based on a system that works"\\
    GP is the unique strategy among AI and machine learning methods - at 2003 - that:
    \stepin
    has duplicated numerous already patented results\\
    has produced patentable new results\\
    has produced "parameterized topology" (point 2. above)
    \stepout
    \stepout
    using Linux CLPs, one might be able to evolve pipe commands that solve something\\
    possible hot applications that jump to my head:
    \stepin
    aesthetics
    \stepin
    maybe there is something deterministic going on in our aesthetic brain parts
    \stepin
    like a neural net, hehe
    \stepin
    then bring back the GP-NN (genetic programming - neural net) loop !!
    \stepin
    i.e. train a NN to be able to judge aesthetics (i.e. make it the fitness measure)\\
    and thus evolve some masterpieces by GP
    \stepout
    \stepout
    \stepout
    \stepout
    NNs (neural nets) themselves
    \stepin
    you know how complicated ChatGPT's language model is? (something about 96-fold loop)\\
    in general, NNs have a predefined structure
    \stepin
    amount of layers, amount of neurons, amount of inputs, amount of outputs, etc.
    \stepout
    let GP create this structure for you\\
    i.e. have a training set ready for each candidate-NN, train it and measure its fitness by how well it predicts
    \stepout
    GP itself - meta!!!
    \stepin
    there's some (albeit little) structure predefined for a GP-evolution
    \stepin
    e.g. nodes and leaves (of the program tree)\\
    e.g. the whole idea behind it (but this might get out-of-hand)
    \stepout
    use GP to predefine this structure
    \stepin
    but this GP needs also nodes and leaves... which one?\\
    well, a candidate can choose to ignore so the more the better (maybe; hopefully)
    \stepout
    \stepout
    math and physics
    \stepin
    of course, I don't yet know what the nodes and leaves should be !!\\
    but the two are areas with some characteristics that are shared across the field of GP-applications
    \stepin
    e.g. we seek a solution to a problem
    \stepin
    this sounds pretty tautological, but I don't think e.g. writing poetry can be phrased this way
    \stepout
    e.g. we might be able to tell automatically if a candidate is a (perfect) solution
    \stepout
    it might be impossible to have a non-discrete/non-binary fitness measure, idk
    \stepin
    not even sure if that breaks the GP-apparatus, though
    \stepout
    so there's a big ? on these examples
    \stepin
    but it would be freaking AMAZING
    \stepout
\end{cascade}
\begin{cascade}
    20240702
    \stepin
    there is criticism on metaphor methodology where new optimization methods try to be sold b.m.o. good metaphors and not high efficiency
    \stepin
    \url{https://en.wikipedia.org/wiki/List_of_metaphor-based_metaheuristics#Criticism}
    \stepout
    Coevolution provides Free Lunch algorithms
    \stepin
    Coevolution is, I think, when two species mutually depend\\
    https://de.wikipedia.org/wiki/Koevolution
    \stepout
    Phenotype vs. Genotype = Problem space vs. Solution space \url{https://en.wikipedia.org/wiki/Genetic_representation}
    \stepin
    Genotype = the set of all genetic code of a candidate\\
    Phenotype = the set of all observable characteristics of a candidate
    \stepin
    e.g. physical form and structure, its developmental processes, its biochemical and physiological properties, its behavior, and the products of behavior\\
    it results from two basic factors
    \stepin
    the expression of an organism's genetic code (its genotype) and the influence of environmental factors
    \stepout
    \stepout
    problem space contains concrete solutions\\
    search space contains the encoded solutions\\
    mapping from search space to problem space is called genotype-phenotype mapping
    \stepin
    genetic operators apply to the search space\\
    for evaluation, elements of the search space are mapped to elements of the problem space via genotype-phenotype mapping
    \stepout
    \stepout
    4 main paradigms in EAs: Genetic Algorithm (GA), Evolution Strategies (ES), Evolutionary Programming (EP), Genetic Programming (GP)
    GA
    \stepin
    focuses on recombination, i.e. crossover, i.e. sex\\
    2 (or more) algorithm-parents are chosen and a new one is created from them
    \stepout
    ES \url{https://en.wikipedia.org/wiki/Evolution_strategy}
    \stepin
    problem space = solution space\\
    dynamic mutation
    \stepin
    it is fixed that each mutation follows a normal distribution\\
    the $\sigma$ of these normal distributions is what is "dynamic"
    \stepin
    it is calculated from the 2 parent's recombined $\sigma$s that is mutated (i.e. changed) via an exponentiated normally distributed random variable
    \stepout
    the new $\sigma$ is then the std. deviation that mutates the actual decision variables of the candidate
    \stepin
    this means that a normally distributed (w/ the new $\sigma$) random variable is added to the old decision variable to give the new decision variable
    \stepout
    \stepout
    upside
    \stepin
    we search with increasing precision as the $\sigma$ may get narrower and narrower
    \stepin
    --> sounds like fractality
    \stepout
    \stepout
    downsides
    \stepin
    danger of getting stuck in larger invalid areas of the search space
    \stepin
    --> fractality may help
    \stepout
    \stepout
    \stepout
    EP
    \stepin
    there's also recombination, but the primary operator is mutation\\
    what is really special is that solutions/candidates are Finite State Machines (FSM)
    \stepin
    they are like programs that have multiple (but finitely many) states between which the transition
    \stepin
    how to transition (for fixed input) is called the FSM's "behaviour"
    \stepin
    this behaviour is what is optimized through mutation (and recombination) in EP
    \stepout
    \stepout
    \stepout
    \stepout
    GP \url{https://en.wikipedia.org/wiki/Genetic_programming}
    \stepin
    there's also mutation, but the primary operator is crossover\\
    it is concerned only with decision trees i.e. actual software programs
    \stepin
    trees: nodes are functions/operations; leaves are variables/constants
    \stepout
    it evolves programs to find the best output to a given input\\
    it recombines subtrees of the parents during crossover\\
    one goal is "algorithm discovery"!
    \stepin
    this sounds pretty much like what I had in mind, but let's keep digging..
    \stepout
    One big guy in this field was John Koza [http://www.genetic-programming.com/johnkoza.html]
    \stepin
    read \url{https://www.popsci.com/scitech/article/2006-04/john-koza-has-built-invention-machine/}, I also saved it as pdf by Keats\\
    his books' videos are at \url{https://human-competitive.org/}
    \stepout
    GECCO: conference on genetic and evolutionary computation \url{https://gecco-2024.sigevo.org/HomePage}
    \stepin
    yearly, starts in 12 days, haha...
    \stepout
    A still-active big guy is Mengjie Zhang at Wellington \url{https://people.wgtn.ac.nz/Mengjie.Zhang}
    \stepin
    he is "available" for PhD supervisions
    \stepout
    The "Humies" are an competition for the best GP-created results that are "better or equal" to human results
    \stepin
    cf. \url{https://human-competitive.org/awards}\\
    there's 8 possible ways how a result can classify as "human-competitive"
\end{cascade}
\begin{cascade}
    20240701
    \stepin
    Ok this is a broad field: Evolutionary Algorithm (EA)
    \stepin
    this is even a Wiki SERIES, cf. \url{https://en.wikipedia.org/wiki/Category:Evolutionary_algorithms}
    \stepin
    "[EA] is a heuristic optimization algorithm using techniques inspired by mechanisms from organic evolution such as mutation, recombination, and natural selection to find an optimal configuration for a specific system within specific constraints."
    \stepout
    but I think this is only about optimization
    \stepin
    I'm still hoping that Evolution is not actually an optimization problem
    \stepin
    e.g. no absolute fitness function in Evolution\\
    e.g. time-dependence in the fitness metric in Evolution
    \stepout
    creation, I think, is qualitatively different from optimization
    \stepout
    Natural Evolution Strategies (NES) seem to treat the fitness function as a black box
    \stepin
    read "High Dimensions and Heavy Tails for Natural Evolution Strategies"
    \stepin
    the heavy tails ought to incorporate also Black Swans, i.e. non-Gaussians
    \stepin
    although I don't know which distribution was the one that was Gaussian in the first place
    \stepout
    \stepout
    \stepout
    No fractality at first sight
    \stepin
    I want to treat different scales on equal footing
    \stepin
    a leave solution + a branch solution + a stem solution = tree solution
\end{cascade}
\begin{cascade}
    20240627
    \stepin
    Bremermann I think wanted to do what I want to do: 
    \stepin
    creation of solutions/optimization to given problems/cost-functions by mimicking evolution
    \stepout
    What I don't want to follow is the strong demand to mimic the biology of evolution
    \stepin
    it's more the guiding principle of evolution that I want to see\\
    abstractly and concisely
    \stepin
    e.g. "survival of the fittest"
    \stepin
    but that's probably not it\\
    sex is missing in it
    \stepout
    \stepout
    maybe I don't even want to mimic evolution but Nature
    \stepin
    fractal-like self-similarity across scales
    \stepin
    read "the Black Swan" by Taleb\\
    look at the Wiki page on the "Mandelbrot set"
    \stepout
    \stepout
    or maybe I want a combination of the two:
    \stepin
    evolutionary randomness\\
    + fractally searching across scales
    \stepin
    for the randomness, i.e. the possible algorithms/creations
    \stepout
    \stepout
    \stepout
    Recursion and fractality are similar
    \stepin
    read \url{https://www.spiceworks.com/tech/devops/articles/what-is-dynamic-programming/}
\end{cascade}
\end{document}
